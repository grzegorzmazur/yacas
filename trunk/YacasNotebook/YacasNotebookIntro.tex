%-*-Yacas-Notebook-*-
\documentclass[12pt]{article}
\input yacas-notebook

\def\ys{\textsf{yacas}}
\def\yn{\textsf{yacas}\texttt{-notebook}}
\begin{document}

\title{A Very Quick Introduction to \yn}
\author{Jay Belanger}
\date{}

\maketitle

\section{Introduction}

\yn\ is an Emacs mode which allows a user to interact with \ys\
while in an Emacs buffer, in a notebook-type way, and allows for the
results to be typeset using \LaTeX.  It is based on Dan
Dill's \TeX/Mathematica.  \yn\ is an extension of
\TeX\ mode, and so has most of the \TeX-mode commands available
(although \texttt{TeX-command-master} has been rebound to 
\texttt{C-c C-c C-c}).  To install \yn, place \texttt{yacas-notebook.el} and
\texttt{yacas.el} somewhere in the load path for Emacs, and put
\texttt{yacas-notebook.tex} somewhere in the \TeX\ inputs path.  Make sure
that \texttt{yacas-notebook.el} is loaded  (say by putting the line
\begin{verbatim}
(autoload 'yacas-notebook-mode "yacas-notebook" "yacas-notebook mode" t)
\end{verbatim}
in your \texttt{.emacs} file) and typing \texttt{M-x yacas-notebook-mode}.  In
order for \LaTeX\ to format the document correctly, you must input
\texttt{yacas-notebook.tex} by writing
\begin{verbatim}
\input yacas-notebook
\end{verbatim}
somewhere in the preamble to the document.
If you want font-locking, then you need to have it turned on (for GNU
Emacs, this can be done by having  
\begin{verbatim}
(global-font-lock-mode 1)
\end{verbatim}
in your \texttt{.emacs} file) and you need
\texttt{font-latex.el} somewhere in
your Emacs load path, this file is included with Auc\TeX. 

\section{Cells}

A cell is the basic unit of \ys\ code in \yn.  It consists of
text between
\begin{verbatim}
\yacas
\end{verbatim}
and
\begin{verbatim}
\endyacas
\end{verbatim}
A cell can be created by typing \texttt{C-c C-c o
  (yacas-notebook-create-cell)}.  (Most of \yn's keybindings begin with
\texttt{C-c C-c}, the \texttt{o} in this case stands for
\textbf{o}pening a cell.)  The delimiters will then be placed in the
buffer, and the point will be placed between them.  While in a cell,
\texttt{yacas.el}'s formatting commands will be available. (The
keybindings will have and extra \texttt{C-c} in front, so they won't
conflict with the \TeX\ keybindings.  For example, in
\texttt{yacas-mode}, to insert an \textbf{If} statement, use
\texttt{C-c C-i}, in \texttt{yacas-notebook-mode}, use \texttt{C-c C-c
C-i}. The formatting commands will always be available; however,
outside of a cell, the indentation may not work correctly.)

When working with several cells, you can jump between them by using
\texttt{C-c C-c + (yacas-notebook-forward-cell)} to go to the next cell and
\texttt{C-c C-c - (yacas-notebook-backward-cell)} to go to the previous cell.

\section{Evaluating Cells}

The contents of a cell can be sent to a buffer in which \ys\ is
running by the command \texttt{C-c C-c s (yacas-notebook-send-cell)}.  (The
\ys\ cell can be killed when desired by the command 
\texttt{C-c C-c k (yacas-kill-job)}.)  While the cell contents will
then be evaluated, by default you won't be able to see the \ys\
buffer.  To see the \ys\ buffer in the bottom half of the window,
you can use the command \texttt{C-c C-c B (yacas-notebook-show-yacas-buffer)},
to hide the \ys\ buffer again, use \texttt{C-c C-c b
(yacas-notebook-dont-show-yacas-buffer)}.  However, \yn\ is designed to
have all the action take place in the document buffer.  To see the
\ys\ output in the document buffer, the command \texttt{M-x
  yacas-notebook-put-output} will take the most recent output in the \ys\
buffer and place it in the \yn\ cell, separated from the input by
the marker
\begin{verbatim}
\output
\end{verbatim}
(Note that if there is more than one command in the cell, only the
output from the last command will be returned, although all the
commands will be evaluated.  This is because only the output from the
last command appears in the \ys\ buffer.)
The command \texttt{C-c C-c u (yacas-notebook-update)} will both send the cell
to the \ys\ buffer and return the results.  To differentiate
$\sin(x^2)$, for example, type \texttt{D(x)Sin(x\^{}2);} in a cell:
\begin{verbatim}
\yacas
D(x)Sin(x^2);
\endyacas
\end{verbatim}
After typing \texttt{C-c C-c u}, it will look like
\begin{verbatim}
\yacas
D(x)Sin(x^2);
\output
           /  2 \
2 * x * Cos\ x  /
\endyacas
\end{verbatim}
(By default, the output will be in \textbf{PrettyForm} from.  This
behavior can be changed by typing \texttt{M-x yacas-toggle-prettyform}.)
To delete the output and return the cell to its original form, you can
use the command \texttt{C-c C-c d (yacas-notebook-delete-output)}.  If the
document is to be \TeX{}ed, the above cell will look like
\yacas
D(x)Sin(x^2);
\endyacas
and the cell with output will look like
\yacas
D(x)Sin(x^2);
\output
           /  2 \
2 * x * Cos\ x  /
\endyacas

\yn\ can take advantage of the fact that \ys\ can give its output
in \TeX\ form.  The command \texttt{C-c C-c U (yacas-tex-update)}
works the same as \texttt{C-c C-c u}, except now the output is in
\TeX\ form, ready to be formatted by \TeX.  In general, if
\texttt{C-c C-c} \textsl{letter} returns \ys\ output, 
\texttt{C-c C-c} \textsl{capital letter} will return the output in
\TeX\ form.  (For the most part, \yn\ should work with either
\TeX\ or \LaTeX\.)  The above cell would become
\begin{verbatim}
\yacas
D(x)Sin(x^2);
\outputtex
2 x \left( \cos x ^{2}\right) 
\endyacas
\end{verbatim}
which, when \TeX{}ed, would become
\newpage
\yacas
D(x)Sin(x^2);
\outputtex
2 x \left( \cos x ^{2}\right) 
\endyacas

(Note that whenever a cell is updated, any old output is discarded and
replaced with new output.)  The command \texttt{C-c C-c a
(yacas-notebook-eval-all)} will update all of the cells in your document,
stopping at each one to ask if you indeed want it updated.  Given an
argument, \texttt{C-u C-c C-c a}, it will update all of the cells in
the document without asking.  The command \texttt{C-c C-c A} behaves
similarly, except now all the output is returned in \TeX\ form.

\section{Initialization Cells}

It is possible that you want certain cells evaluated separately from
the others; perhaps, for example, you want certain cells evaluated
whenever you open the document.  This can be done using initialization
cells.  An initialization cell is delimited by
\begin{verbatim}
\yacas[* Initialization Cell *]

\endyacas
\end{verbatim}
The command \texttt{C-c C-c q (yacas-notebook-toggle-init)} will turn a cell
into an initialization cell, applying \texttt{C-c C-c q} again will
turn it back into a regular cell.  When \TeX{}ed, an initialization
cell will look like
\yacas[* Initialization Cell *]
D(x)Sin(x^2);
\endyacas

Initialization cells behave like regular cells, except that they can
be evaluated as a group.  To evaluate all initialization cells
(without displaying the output in the document buffer), the command
\texttt{C-c C-c Q (yacas-notebook-eval-init)} will go to each of the
initialization cells and evaluate them.  If you want the output of the
cells to be brought back, stopping at each cell to see if you want it
updated, then use the command \texttt{C-c C-c v
  (yacas-notebook-update-init)}.  With an argument, \texttt{C-u C-c C-c v},
the initialization cells will be updated without asking  The command
\texttt{C-c C-c V (yacas-notebook-tex-eval-init)} behaves just like
\texttt{C-c C-c c}, except that the output is returned in \TeX\ form.

\section{Referencing other cells}

Along \ys\ code, a cell can contain references to other cells, and
when the original cell is sent to \ys, the reference is replaced by
the referenced cell's contents (but only in the \ys\ buffer, the
cell's contents in the document buffer is not changed).  In order to
do this, the original cell must be marked by having a label of the
form \texttt{<}\textit{filename:cell label}\texttt{>}.  (The reason
for the \textit{filename} will become apparent later, and \textit{cell
  label} is optional for the referencing cell.)  The referenced cell
must also be labeled, with the same \textit{filename} but a unique
\textit{cell label}.  To reference the other cell, the original cell
need only contain the marker for the referenced cell.  For example,
given cell 1:
\begin{verbatim}
\yacas<filename:optional>
<filename:definef>
D(x)f(x);
\endyacas
\end{verbatim}
and cell 2:
\begin{verbatim}
\yacas<filename:definef>
f(x):=Sin(x^2);
\endyacas
\end{verbatim}
then the result of updating cell 1 (\texttt{C-c C-c u}) will be:
\begin{verbatim}
\yacas<filename:optional>
<filename:definef>
D(x)f(x);
\output
           /  2 \
2 * x * Cos\ x  /
\endyacas
\end{verbatim}
When \TeX{}ed, the top line will contain a copy of the marker:
\newpage
\yacas<filename:optional>
<filename:definef>
D(x)f(x);
\output
           /  2 \
2 * x * Cos\ x  /
\endyacas

A cell can contain more than one reference, and referenced cells can
themselves contain references.

If you want the references in a cell to be replaced by the actual
code, the command \texttt{C-c C-c = (yacas-notebook-assemble-cell)} will
expand all the references and put the code into a separate buffer (so
it will not affect the original document).

\section{Web}


The reason for the ability to reference other cells isso that you can
write what Donald Knuth calls literate programs.  The idea is that the
program is written in a form natural to the author rather than natural
to the computer.  (Another aspect of Knuth's system is that the code
is carefully documented, hence the name ``literate programming'', but
that is done naturally in \yn.) Knuth called his original literate
programming tool \texttt{WEB}, since, as he puts it, ``the structure
of a software program may be thought of as a web that is made up of
many interconnected pieces.''  \yn's ability in this respect is
taken directly from \TeX/Mathematica, and is ultimately based on
\texttt{WEB}.  To create a program, the ``base cell'' or ``package
cell'' should contain a label of the form
\texttt{<}\textit{filename:}\texttt{>} (no cell label), and can
contain references of the form
\texttt{<}\textit{filename:part}\texttt{>} (same file name as the base
cell).  

As a simple example, suppose I want to create a program to sum the
first $n$ squares.  I could start:
\newpage
\begin{verbatim}
\yacas<squaresum:>
SumOfSquares(n) :=
[
   Local(L);
   <squaresum:makelist>
   <squaresum:squarelist>
   <squaresum:addlist>
]; /* SumOfSquares(n) */
\endyacas
\end{verbatim}
I would then need cells:
\begin{verbatim}
\yacas<squaresum:makelist>
L:= 1 .. n;
\endyacas

\yacas<squaresum:squarelist>
<squaresum:definesquare>
L:= MapSingle("square", L);
\endyacas

\yacas<squaresum:addlist>
Sum(L);
\endyacas
\end{verbatim}
and then also
\begin{verbatim}
\yacas<squaresum:definesquare>
square(x) := x^2;
\endyacas
\end{verbatim}

\newpage

When \TeX{}ed, the header of the cell will say that it determines the
file \texttt{squaresum}.
\yacas<squaresum:>
SumOfSquares(n) :=
[
   Local(L);
   <squaresum:makelist>
   <squaresum:squarelist>
   <squaresum:addlist>
]; /* SumOfSquares(n) */
\endyacas

The command \texttt{C-c C-c (yacas-notebook-assemble-package)} will put all
the pieces together in the file it determines.  The resulting file, in
this case, will be named \texttt{squaresum} and will contain:
\begin{verbatim}
SumOfSquares(n) :=
[
   Local(L);
   L:= 1 .. n;
   square(x) := x^2;
   L:= MapSingle("square", L);
   Sum(L);
]; /* SumOfSquares(n) */
\end{verbatim}

\section{Miscellaneous}

It is possible that you want to evaluate part of your document that
isn't in a cell, whether it be a portion of a cell or not in a cell.
The command \texttt{C-c C-c C-r (yacas-region)} will send the current
region to the \ys\ buffer.  The command \texttt{C-c C-c n
  (yacas-notebook-replace-line)}, however, will send the current line to \ys\,
comment out the current line, and the write the \ys\ output in the
current buffer.  The command \texttt{C-c C-c N} will do the same
thing, except it will return the output in \TeX\ form.

\appendix


\section{\yn\ commands}


\vskip .5 in

\begin{tabular}{ll}
\texttt{C-c C-c o} & Create a cell \\
\texttt{C-c C-c u} & Update the cell's output \\
\texttt{C-c C-c U} & Update the cell's output in \TeX\ form \\
\texttt{C-c C-c d} & Delete the cell's output \\
\texttt{C-c C-c +} & Go to the next cell \\
\texttt{C-c C-c -} & Go to the previous cell\\
\texttt{C-c C-c a} & Update all of the cells \\
\texttt{C-u C-c C-c a} & Update all of the cells without prompting\\
\texttt{C-c C-c A} & Update all of the cells in \TeX\ form\\
\texttt{C-u C-c C-c A} & 
                    Update all of the cells in \TeX\ form without prompting\\
\texttt{C-c C-c q} & Toggle whether or not the current cell is an
                 initialization cell\\
\texttt{C-c C-c Q} & Evaluate all of the initialization cells \\
\texttt{C-c C-c v} & Update all of the initialization cells \\
\texttt{C-c C-c V} & Update all of the initialization cells in \TeX form\\
\texttt{C-c C-c =} & Assemble a cell with references \\
\texttt{C-c C-c @} & Assemble a cell which defines a package \\
\texttt{C-c C-c B} & Show the \ys\ buffer \\
\texttt{C-c C-c b} & Don't show the \ys\ buffer\\
\texttt{C-c C-c n} & Replace the current line with \ys\ output \\
\texttt{C-c C-c N} & Replace the current line with \ys\ output in \TeX\ form\\
\end{tabular}

\newpage


\section{\ys\ functions}

Here are the functions from \ys\ mode which are also used in \yn\
mode.

\vskip .5 in

\begin{tabular}{ll}
\texttt{C-c C-c C-r} & Send the current region to \ys\ \\
\texttt{C-c C-c C-k} & Kill the \ys\ process\\
\texttt{C-c C-c C-p} & Insert a procedure \\
\texttt{C-c C-c C-v} & Insert another local variable in a procedure \\
\texttt{C-c C-c C-i} & Insert an \texttt{If} statement\\
\texttt{C-c C-c C-e} & Insert the \texttt{else} part of \texttt{If}\\
\texttt{C-c C-c C-f} & Insert a \texttt{For} statement \\
\texttt{C-c C-c C-a} & Insert a \texttt{ForEach} statement \\
\texttt{C-c C-c C-w} & Insert a \texttt{While} statement\\
\texttt{C-c C-c C-u} & Insert an \texttt{Until} statement\\
\texttt{C-c C-c C-o} & Copy the last \ys\ output\\
\texttt{C-c C-c C-s} & Copy the complete last \ys\ output\\
\texttt{C-c C-c C-t} & Recenter the \ys\ buffer\\
\texttt{C-c C-c \#} & Insert an inline comment \\
\texttt{C-c C-c ]} & Indent a region\\
\texttt{C-c C-c >} & Move forward to the same indentation level \\
\texttt{C-c C-c <} & Move backward to the same indentation level \\
\end{tabular}


\end{document}