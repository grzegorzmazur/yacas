\documentclass{llncs}
\usepackage{graphicx}
\usepackage{amssymb}
%\usepackage[T1]{fontenc}
%\usepackage[latin1]{inputenc}
%\usepackage{babel}

\begin{document}
\title{Computing the incomplete Gamma function to arbitrary precision}

\author{Serge Winitzki\inst{1}}
\institute{Department of Physics, Ludwig-Maximilians University, Theresienstr.~37, 80333 Munich, Germany
(\email{serge@theorie.physik.uni-muenchen.de})
}

\maketitle

\begin{abstract}
I consider an arbitrary-precision computation of the incomplete Gamma
function from the Legendre continued fraction. Using the method of
generating functions, I compute the convergence rate of the continued
fraction and find a direct estimate of the necessary number of terms.
This allows to compare the performance of the continued fraction and
of the power series methods. As an application, I show that the incomplete
Gamma function $\Gamma \left(a,z\right)$ can be computed to $P$
digits in at most $O\left(P\right)$ long multiplications uniformly
in $z$ for $Re\, z>0$. The error function of the real argument,
$\textrm{erf}\, x$, requires at most $O(P^{2/3})$ long multiplications.
\end{abstract}

\section{Introduction}

The incomplete Gamma function (see e.g.~\cite{AS64} Sec.~6.5)\begin{equation}
\Gamma \left(a,z\right)=\int _{z}^{+\infty }t^{a-1}e^{-t}dt\end{equation}
 is a generalization of such special functions as the Gamma function
$\Gamma \left(a\right)$, the exponential integral $\textrm{Ei}\, x$,
and the complementary error function $\textrm{erfc}\, z$. After an
appropriate analytic continuation and branch cuts one obtains a definition
of $\Gamma \left(a,z\right)$ valid for arbitrary complex $a$, $z$.

I consider the general problem of computing $\Gamma \left(a,z\right)$
to a precision of $P$ (decimal) digits. Several methods have been
described in the literature for the computation of the incomplete
Gamma function \cite{Gautschi79,DM86,Gautschi99,Smith01}. However,
these methods were not analyzed in view of a computation at an asymptotically
high precision. In particular, the convergence rate of the Legendre
continued fraction has only been checked heuristically or numerically
for a particular target precision. 

The purpose of this paper is to present a calculation of the convergence
rate of the continued fraction of Legendre and to perform a comparative
analysis of efficiency of the existing methods for an asymptotically
large target precision $P$.


\section{Overview of methods}


\subsection{Mathematical properties}

The incomplete Gamma function admits two series expansions,\begin{equation}
\Gamma \left(a,z\right)=\Gamma \left(a\right)-\sum _{n=0}^{\infty }\frac{\Gamma \left(a\right)e^{-z}z^{n+a}}{\Gamma \left(a+n+1\right)}=\Gamma \left(a\right)-\sum _{n=0}^{\infty }\frac{\left(-1\right)^{n}z^{n+a}}{\left(a+n\right)n!},\label{eq:Gam-ser1}\end{equation}
 an asymptotic series for large $\left|z\right|$,\begin{equation}
\Gamma \left(a,z\right)=z^{a-1}e^{-z}\sum _{n=0}^{\infty }\frac{\Gamma \left(a\right)}{\Gamma \left(a-n\right)z^{n}},\label{eq:Gam-ser2}\end{equation}
 a continued fraction expansion due to Legendre,\begin{equation}
\Gamma \left(a,z\right)=\frac{e^{-z}z^{a}}{z+}\frac{1-a}{1+}\frac{1}{z+}\frac{2-a}{1+}\frac{2}{z+}...,\end{equation}
 and a recurrence relation,\begin{equation}
\Gamma \left(a+n+1,z\right)=\Gamma \left(a+n,z\right)+\frac{\Gamma \left(a\right)e^{-z}z^{n+a}}{\Gamma \left(a+n+1\right)},\quad n\geq 0.\end{equation}
 

Both series of Eq.~(\ref{eq:Gam-ser1}) converge unless $a$ is a
negative integer. The recurrence relation was analyzed in \cite{Gautschi99}
and found to be of limited use for general $a$ and $z$ because of
round-off errors, but usable if $a$ is close to a negative integer.
Another asymptotic expansion due to Temme \cite{Temme79} is cumbersome
and requires costly computations of the Bernoulli numbers. Therefore
we shall focus on the series expansions of Eqs.~(\ref{eq:Gam-ser1})-(\ref{eq:Gam-ser2})
and on the continued fraction. (The asymptotic series will be efficient
when $\left|z\right|\gg P$.)

The error function is related to the incomplete Gamma function by
$\sqrt{\pi }\, \textrm{erfc}\, z$$=\Gamma \left(1/2,z^{2}\right)$
and has similar representations, in particular, the continued fraction
due to Laplace,\begin{equation}
\sqrt{\pi }xe^{x^{2}}\textrm{erfc}\, x=\frac{1}{1+}\frac{v}{1+}\frac{2v}{1+}\frac{3v}{1+}...,\label{eq:erfc-cf}\end{equation}
where $v\equiv \left(2x^{2}\right)^{-1}$. This representation was
used e.g.~in \cite{Thacher63}. Our analysis of the continued fraction
method will include Eq.~(\ref{eq:erfc-cf}) as a particular case.


\subsection{Computation of power series}

A power series $\sum _{k=1}^{n}b_{k}x^{k}$, where all $b_{k}$ are
rational numbers and $x$ is a floating-point number with $P$ digits,
may be computed using $O\left(\sqrt{n}\right)$ long multiplications
using the {}``rectangular'' or {}``baby step/giant step'' method
(see e.g.~\cite{Smith89}). Power series of this form are most often
encountered in approximations of analytic functions. The increased
asymptotic speed of the {}``rectangular'' method is obtained when
the ratios $b_{k}/b_{k-1}$ are rational numbers with {}``short''
numerators and denominators with $O\left(\ln k\right)$ digits. If
these ratios are not short, the best method
is the Horner scheme requiring $O\left(n\right)$ long multiplications.

The series of Eqs.~(\ref{eq:Gam-ser1}), (\ref{eq:Gam-ser2}) have
rational coefficients only when $a$ is a short rational number, e.g.~$a=1/2$
for the error function. If both $a$ and $z$ are short rational numbers,
then the series may be computed using the binary splitting method
\cite{HP98}. This method requires $O\left(\ln n\right)$ multiplications
of $O\left(n\ln n\right)$-digit integers and is asymptotically the
fastest, albeit with limited applicability.


\subsection{Computation of continued fractions}

There are two main classes of methods for the numerical evaluation
of continued fractions: the forward or the backward recurrences (see
e.g.~\cite{PTVF92} for details of various methods). The backward
recurrences are more straightforward but one needs to know the necessary
number of terms $n$ at the outset. The forward recurrences are significantly
slower than the backward recurrences, but do not require to set the
number of terms in advance. In our case, the number of terms can be
estimated analytically and therefore the backward recurrences are
preferable. The backward recurrences also have much better round-off
error behavior \cite{JT74}. The asymptotic complexity of all methods
is $O\left(n\right)$ long multiplications.

Denote by $F_{m,n}$ the partial fraction \begin{equation}
F_{m,n}=a_{m}+\frac{b_{m}}{a_{m+1}+}\frac{b_{m+1}}{a_{m+2}+}...\frac{b_{n-1}}{a_{n}}.\end{equation}
There are two possible backward recurrences for $F_{m,n}$. First,
one may compute the partial fractions directly,\begin{equation}
F_{m,n}=a_{m}+\frac{b_{m}}{F_{m+1,n}},\end{equation}
starting from $F_{n,n}=a_{n}$. This uses $n$ long divisions. Second,
one may compute the numerators $p_{m,n}$ and the denominators $q_{m,n}$
as separate sequences,\begin{equation}
p_{m,n}=a_{m}p_{m+1,n}+b_{m}q_{m+1,n},\quad q_{m,n}=p_{m+1,n},\quad \left(\frac{p_{m,n}}{q_{m,n}}\equiv F_{m,n}\right)\end{equation}
 with $p_{n,n}=a_{n}$, $q_{n,n}=1$, and perform the final division
$F_{0,n}=p_{0,n}/q_{0,n}$. This requires $2n$ long multiplications.
Since a division usually takes at least four times longer than a multiplication
\cite{KM97}, the latter method is faster at high precision.

A refinement of the backward recurrence method \cite{Ts88} is to
use an ansatz for $F_{n,n}$ that more closely approximates the infinite
remainder $F_{n,\infty }$ of the continued fraction. The ansatz follows
from the assumption that $F_{n,\infty }$ changes slowly with $n$
at large $n$; then one finds $F_{n,\infty }$ from the equation\begin{equation}
F_{n,\infty }=a_{n}+\frac{b_{n}}{F_{n,\infty }}.\end{equation}
Sometimes it is better to use instead the equation\begin{equation}
F_{n,\infty }=a_{n}+\frac{b_{n}}{a_{n+1}+\frac{b_{n+1}}{F_{n,\infty }}}.\label{eq:Frem-est}\end{equation}
The validity of the ansatz for $F_{n,\infty }$ needs to be checked
in each particular case. Using an ansatz for $F_{n,\infty }$ does
not give an asymptotic improvement of speed, but the precision of
the approximation is typically improved by several orders. One could
also obtain $F_{n,\infty }$ more rigorously as a series in $n^{-1}$,
but numerical tests suggest that this does not give a significant
advantage over Eq.~(\ref{eq:Frem-est}).

Finally, when both $a$ and $z$ are short rational numbers, the binary
splitting technique can be used to evaluate the continued fraction
exactly (as a rational number). This would require $O\left(\ln n\right)$
multiplications of $O\left(n\ln n\right)$-digit integers.


\section{Convergence of Legendre's continued fraction}

The continued fraction of Legendre can be rewritten similarly to Eq.~(\ref{eq:erfc-cf}),\begin{equation}
e^{z}z^{1-a}\Gamma \left(a,z\right)=\frac{1}{1+}\frac{\left(1-a\right)v}{1+}\frac{v}{1+}\frac{\left(2-a\right)v}{1+}\frac{2v}{1+}...,\label{eq:iG-frac}\end{equation}
where $v\equiv z^{-1}$. The terms are $a_{0}=0$, $b_{0}=1$, $a_{n}=1$,
$b_{2n-1}=\left(n-a\right)v$, $b_{2n}=nv$ (for $n\geq 1$). This
continued fraction converges for all $z$ except real $z\leq 0$;
however, the speed of convergence varies with $z$. Below I assume
that $a$ is not a positive integer (or else the continued fraction
is finite).

Analytic estimates of the convergence rate of continued fractions
are not often found in the literature. The estimate in Ref.~\cite{Gautschi79}
is partly heuristic and is only valid for real $a$. There is an analytic
estimate for the continued fraction for Dawson's integral \cite{McCabe74},
but that estimate is based on a special property that does not hold
for $\Gamma \left(a,z\right)$. Several general error estimates are
available \cite{Field77} but they require long computations. In this
section I use the particular form of the Legendre continued fraction
of Eq.~(\ref{eq:iG-frac}) to obtain a direct \emph{a priori} estimate
of the required number of terms for a given absolute precision.

The forward recurrence\begin{equation}
F_{0,n}-F_{0,n-1}=\frac{\left(-1\right)^{n}b_{0}...b_{n-1}}{Q_{n}Q_{n+1}},\end{equation}
where the sequence $Q_{n}$ is defined by $Q_{0}=0$, $Q_{1}=1$,
$Q_{n+2}=a_{n+1}Q_{n+1}+b_{n}Q_{n}$, gives an estimate for the convergence
rate, provided that we have an asymptotic expression for the growth
of $Q_{n}$. We shall use an analogous expression for the staggered
convergents,\begin{equation}
F_{0,n}-F_{0,n-2}=\frac{\left(-1\right)^{n-1}a_{n}b_{0}...b_{n-2}}{Q_{n-1}Q_{n+1}}.\label{eq:F2diff}\end{equation}


The sequence $Q_{n}$ for the continued fraction of Eq.~(\ref{eq:iG-frac})
is defined by\begin{eqnarray}
Q_{2n+2} & = & Q_{2n+1}+nvQ_{2n},\label{eq:Qeven}\\
Q_{2n+1} & = & Q_{2n}+\left(n-a\right)vQ_{2n-1}.\label{eq:Qodd}
\end{eqnarray}
Equation~(\ref{eq:F2diff}) gives\[
F_{0,2n}-F_{0,2n-2}=-\frac{\Gamma \left(n-a\right)v^{2n-2}\left(n-1\right)!}{\Gamma \left(1-a\right)Q_{2n-1}Q_{2n+1}}.\]
 We now need to estimate the asymptotic growth of $Q_{n}$. Since
the recurrences for the odd and the even terms $Q_{n}$ are different,
we are motivated to consider a pair of generating functions\begin{equation}
F\left(s\right)=\sum _{n=0}^{\infty }Q_{2n}\frac{s^{n}}{n!},\quad G\left(s\right)=\sum _{n=0}^{\infty }Q_{2n+1}\frac{s^{n}}{n!}.\end{equation}
Using Eqs.~(\ref{eq:Qeven})-(\ref{eq:Qodd}), it is easy to show
that\begin{equation}
G=\left(1-vs\right)\frac{dF}{ds},\quad \frac{dF}{ds}=\left(1-vs\right)\frac{dG}{ds}-\left(1-a\right)vG.\end{equation}
With the initial condition $G\left(0\right)=1$, one obtains\begin{equation}
G\left(s\right)=\left(1-vs\right)^{a-1}\exp \left[\frac{s}{1-vs}\right].\end{equation}
The asymptotic behavior of $Q_{2n-1}$ for large $n$ can be found
by the method of steepest descents (see e.g.~\cite{Olver74}) on
the contour integral around $s=0$,\begin{equation}
Q_{2n-1}=\frac{\left(n-1\right)!}{2\pi i}\oint _{\left(0\right)^{+}}G\left(s\right)s^{-n}ds.\label{eq:Q-oint}\end{equation}
The saddle points $s_{1,2}$ are found from the (quadratic) equation\begin{equation}
\frac{dg}{ds}=-\frac{n}{s}-v\frac{a-1}{1-vs}+\frac{1}{\left(1-vs\right)^{2}}=0,\end{equation}
where $g\left(s\right)\equiv \ln \left[G\left(s\right)s^{-n}\right]$.
We are only interested in the leading asymptotic of the result for
large $n$. The contribution of a saddle point $s_{i}$ to Eq.~(\ref{eq:Q-oint})
is\begin{equation}
\frac{2\pi iQ_{2n-1}}{\left(n-1\right)!}\sim \frac{\sqrt{2\pi }e^{g\left(s_{i}\right)}}{\sqrt{-g''\left(s_{i}\right)}}=i\sqrt{\pi n}\frac{\exp \left(\pm 2\sqrt{nz}-\frac{z}{2}\right)}{z^{n}\left(\pm \sqrt{\frac{n}{z}}\right)^{a+\frac{3}{2}}}\left(1+O\left(n^{-1/2}\right)\right).\end{equation}
We find (using the standard branch cut $Re\, \sqrt{z}\geq 0$) that
the point $s_{1}$ (upper signs) gives the dominant contribution unless
$Re\, \sqrt{v}=0$. Then we obtain\begin{equation}
\left|F_{0,2n}-F_{0,2n-2}\right|=\frac{4\pi \left|z^{\frac{3}{2}-a}\right|\sqrt{n}}{\left|\Gamma \left(1-a\right)\right|}\left|\exp \left(-4\sqrt{nz}+z\right)\right|\left(1+O\left(n^{-1/2}\right)\right).\label{eq:Fans}\end{equation}
This asymptotic is valid for $n\gg \left|z\right|$, $\left|a\right|$.
With our branch cut for $\sqrt{z}$, \begin{equation}
\left|\exp \left(-\sqrt{z}\right)\right|=\exp \left(-\sqrt{\frac{\left|z\right|+Re\, z}{2}}\right),\end{equation}
and it is clear that the sequence of Eq.~(\ref{eq:Fans}) decays
with $n$ unless $z$ is real and $z\leq 0$. The number of terms
$n$ needed to achieve an absolute precision $\varepsilon =10^{-P}$
can be estimated from Eq.~(\ref{eq:Fans}); one finds that $n$ satisfies
the equation\begin{equation}
\sqrt{n}=\frac{P\ln 10+\ln \left(4\pi \sqrt{n}\right)+Re\, \left[z+\left(\frac{3}{2}-a\right)\ln z-\ln \Gamma \left(1-a\right)\right]}{\sqrt{8\left(\left|z\right|+Re\, z\right)}}.\label{eq:n-iter}\end{equation}
To find $n$, it is enough to perform a few iterations of Eq.~(\ref{eq:n-iter})
starting with $n=1$. It is clear that at fixed $z$ one needs $2n=O\left(P^{2}\right)$
terms of the continued fraction for $P$ digits of precision; the
constant of proportionality depends on $a$ and $z$. The required
number of terms grows when $z$ approaches the negative real semiaxis
or zero, or when $Re\, a>0$ and $\left|a-1\right|>e\left|z\right|$.

Numerical tests show that the estimate of Eq.~(\ref{eq:Frem-est}),\begin{equation}
F_{2n,\infty }\approx \frac{1}{2}\left(1+\frac{a-1}{z}+\sqrt{\frac{4n}{z}+\left(1-\frac{a-1}{z}\right)^{2}}\right),\end{equation}
 gives a good approximation to the infinite remainder $F_{2n,\infty }$
of the continued fraction of Eq.~(\ref{eq:iG-frac}) for large $n$.
Use of this estimate improves the resulting precision by a few orders
of magnitude but the convergence rate is unchanged.

The same analysis holds for the continued fraction of Eq.~(\ref{eq:erfc-cf})
which is a special case of Eq.~(\ref{eq:iG-frac}) with $a=1/2$.


\section{Comparison of methods}


\subsection{Convergence of the Taylor series}

Both series of Eq.~(\ref{eq:Gam-ser1}) converge for all $a$ and
$z$ (except when $a$ is a negative integer), but their convergence
and round-off behavior are not uniform. The series are not efficient
when $a$ is equal or close to a negative integer because of a cancellation
between $\Gamma \left(a\right)$ and some terms of the form $1/(a+n)$.
In this case, the recurrence relation can be used \cite{Gautschi99}.

Consider the first series of Eq.~(\ref{eq:Gam-ser1}). The magnitude
of the ratio of successive terms is $\left|z\right|/\left|a+n+1\right|$.
If $Re\, \left(a+1\right)>0$ and $\left|a+1\right|>\left|z\right|$,
or if $|Im\, a|>\left|z\right|$, then $\left|z\right|<\left|a+n+1\right|$
for all $n$ and the convergence is monotonic. In this case it is
enough to take $O\left(P/\ln P\right)$ terms for a precision of $P$
digits. Otherwise, the rapid convergence regime begins only after
$n_{0}=O\left(\left|z\right|\right)$ terms. More precisely, $n_{0}$
is the largest positive integer for which $\left|a+n_{0}\right|<\left|z\right|$;
approximately, \begin{equation}
n_{0}\approx -Re\, a+\sqrt{\left|z\right|^{2}-\left(Im\, a\right)^{2}}.\label{eq:n01est}\end{equation}
 The sum of the first $n_{0}$ terms may involve cancellations (especially
for $Re\, z<0$), and the working precision may need to be increased
by $O\left(P\right)$ digits. This will not change the asymptotic
complexity of the calculation. Therefore, we may estimate the complexity
of the first series by $O\left(P/\ln P\right)+O\left(\left|z\right|\right)$
long multiplications.

The second series of Eq.~(\ref{eq:Gam-ser1}) is simpler to analyze.
The convergence is monotonic when $\left|z\right|<1$, while for $\left|z\right|>1$
one needs $n_{0}=O\left(\left|z\right|\right)$ additional terms until
the rapid convergence is achieved. (Again, we assume that $a$ is
not close to a negative integer.) Unlike the first series, the largest
cancellations occur with $Re\, z>0$. We obtain the same asymptotic
complexity of $O\left(P/\ln P\right)+O\left(\left|z\right|\right)$
long multiplications.


\subsection{Convergence of the asymptotic series}

The asymptotic series of Eq.~(\ref{eq:Gam-ser2}) starts to diverge
after the $n_{0}$-th term,\begin{equation}
n_{0}\approx Re\, a+\sqrt{\left|z\right|^{2}-\left(Im\, a\right)^{2}},\end{equation}
 when the ratio of the successive terms $\left|n-a\right|/\left|z\right|$
becomes large. Therefore, Eq.~(\ref{eq:Gam-ser2}) can be used for
calculation if $\left|a\right|<\left|z\right|$, or if $\left|Im\, a\right|<\left|z\right|$
and $Re\, a>0$. The minimum absolute error of the series is of the
order of the $n_{0}$-th term,\begin{equation}
\left|z^{a-1}e^{-z}\frac{\Gamma \left(a\right)z^{-n_{0}}}{\Gamma \left(a-n_{0}\right)}\right|\sim e^{-2n_{0}}\end{equation}
(assuming $\left|z\right|>\left|a\right|$). This error is below the
required precision $\varepsilon =10^{-P}$ when $2n_{0}\sim 2\left|z\right|>P\ln 10$.
The working precision needs to be increased by $O\left(P\right)$
digits. We find that the cost of computation for a precision $P$
for $\left|z\right|\gg P$ is at most $O\left(P\right)$ multiplications.


\subsection{Complexity of $\Gamma \left(a,z\right)$}

It is usual that the series and the continued fraction methods have
somewhat complementary domains of applicability. The series converge
rapidly at small $z$, while the continued fraction is most efficient
at $1<\left|a\right|<\left|z\right|$ or when $Re\, a<0$. Using the
explicit estimates of Eqs.~(\ref{eq:n-iter}) and (\ref{eq:n01est}),
one can find the best methods. 

For the practically important case $Re\, z>0$, one could choose the
series method for $\left|z\right|<P$ (which requires at most $O\left(P\right)$
terms) and the continued fraction method for $\left|z\right|>P$ (using
again $O\left(P^{2}/z\right)\sim O\left(P\right)$ terms) if $Re\, a<0$
and $\left|a\right|<\left|z\right|$. The asymptotic series method
is used for $\left|z\right|>P$, $Re\, a>0$ and $\left|z\right|>\left|a\right|$.
Thus, the computation of the incomplete Gamma function $\Gamma \left(a,z\right)$
for $Re\, z>0$ requires $O\left(P\right)$ long multiplications uniformly
in $z$.

As already noted in \cite{Gautschi79}, the computation of $\Gamma \left(a,z\right)$
for $Re\, z<0$ is more computationally intensive. At the moment,
a uniform bound on complexity does not seem to be available, although
at fixed $z$ the complexity is $O\left(P\right)$ if any of the series
of Eq.~(\ref{eq:Gam-ser1}) is used.


\subsection{Complexity of $\textrm{erfc}\, x$}

The complementary error function $\textrm{erfc}\, x$ can be computed
somewhat faster than $\Gamma \left(a,z\right)$ because the {}``rectangular''
method can be applied to the series of Eq.~(\ref{eq:Gam-ser1}),
reducing their complexity to $O\left(\sqrt{P/\ln P+O\left(\left|z\right|\right)}\right)$
long multiplications (here $z=x^{2}$). If we use the series for real
$x$ such that $\left|x\right|<O\left(P^{2/3}\right)$ and the continued
fraction for larger $\left|x\right|$, the overall complexity becomes
$O\left(P^{2/3}\right)$ uniformly in $x$.


\subsection{Rational arguments}

If both $a$ and $z$ are {}``short'' rational numbers with $O\left(1\right)$
digits in the numerators and the denominators, the binary splitting
technique can be used for the series as well as for the continued
fraction computations (see Appendix). The complexity of the binary
splitting method at very high precision can be estimated as $O\left(\ln P\right)$
multiplications of $O\left(P\ln P\right)$-digit integers. As before,
this estimate is uniform in the argument $z$ if $Re\, z>0$.

\appendix

\section{The binary splitting technique for continued fractions}

If a continued fraction\begin{equation}
F_{0,n}=a_{0}+\frac{b_{0}}{a_{1}+}\frac{b_{1}}{a_{2}+}...\frac{b_{n-2}}{a_{n-1}+}\frac{b_{n-1}}{a_{n}}\label{eq:F0ndef}\end{equation}
contains only {}``short'' terms, i.e.~$a_{n}$ and $b_{n}$ are
$O\left(\ln n\right)$-digit integers or rationals, then the binary
splitting technique can be applied to the computation of $F_{0,n}$.
Here we sketch the algorithm and show that its complexity is equivalent
to $O\left(\ln n\right)$ multiplications of $O\left(n\ln n\right)$-digit
integers.

Consider Eq.~(\ref{eq:F0ndef}) as a function of $a_{n}\equiv p_{n}/q_{n}$,
and compute the coefficients $A\left(0,n\right)$, $B\left(0,n\right)$,
$C\left(0,n\right)$, $D\left(0,n\right)$ of the equivalent projective
transformation for $\left(p_{n},q_{n}\right)$:\begin{equation}
F_{0,n}=\frac{A\left(0,n-1\right)a_{n}+B\left(0,n-1\right)}{C\left(0,n-1\right)a_{n}+D\left(0,n-1\right)}=\frac{A\left(0,n-1\right)p_{n}+B\left(0,n-1\right)q_{n}}{C\left(0,n-1\right)p_{n}+D\left(0,n-1\right)q_{n}}.\end{equation}
When we compute these coefficients as exact integers, we would only
need to substitute the desired value for $a_{n}$ to find $F_{0,n}$.

The coefficients $A$, $B$, $C$, $D$ are computed for the subintervals
$\left(0,n'-1\right)$ and $\left(n',n-1\right)$ recursively, using
the matrix product of projective transformations,\begin{equation}
T\equiv \left(\begin{array}{cc}
 A & B\\
 C & D\end{array}
\right);\quad T\left(0,n-1\right)=T\left(0,n'-1\right)T\left(n',n-1\right).\end{equation}
Each time $n'$ is chosen near the middle of the interval $\left(0,\, n-1\right)$.
At the base of recursion,\begin{equation}
T\left(k,k\right)=\left(\begin{array}{cc}
 a_{k} & b_{k}\\
 1 & 0\end{array}
\right).\end{equation}


The complexity of this method can be found as follows. The depth of
the recursion is $O\left(\ln n\right)$ and there are $2^{k}$ matrix
multiplications at level $k$. The elements of the matrix $T\left(l,m\right)$
have $O\left(\left(m-l\right)\ln m\right)$ digits, and therefore
the matrices at level $k$ have elements with $O\left(n2^{-k}\ln n\right)$
digits. Since a multiplication is at least linear in the number of
digits, $2^{k}$ multiplications with $n2^{-k}$ digits are faster
than one multiplication with $n$ digits. Therefore the total number
of equivalent $O(n\ln n)$-digit multiplications is equal to the recursion
depth $O\left(\ln n\right)$.

\begin{thebibliography}{10}
\bibitem{AS64}M. Abramowitz and I. Stegun, eds., \emph{Handbook of special functions},
National Bureau of Standards, 1964.
\bibitem{Gautschi79}W. Gautschi, ACM TOMS \textbf{5}, p. 466 (1979); \emph{ibid.}, {}``Algorithm
542'', p. 482.
\bibitem{DM86}A. R. DiDonato and A. H. Morris, Jr., ACM TOMS \textbf{12}, p. 377
(1986).
\bibitem{Gautschi99}W. Gautschi, ACM TOMS \textbf{25}, p. 101 (1999).
\bibitem{Smith01}D. M. Smith, {}``Algorithm 814'', ACM TOMS \textbf{27}, p. 377 (2001).
\bibitem{Temme79}N. M. Temme, SIAM J. Math.~Anal.~\textbf{10}, p. 757 (1979).
\bibitem{Thacher63}H. C. Thacher, Jr., {}``Algorithm 180'', Comm.~ACM \textbf{6},
p. 314 (1963).
\bibitem{Smith89}D. M. Smith, Math. Comp. \textbf{52}, p. 131 (1989).
\bibitem{HP98}B. Haible and T. Papanikolaou, LNCS \textbf{1423}, p. 338 (Springer,
1998).
\bibitem{PTVF92}W. H. Press, S. A. Teukolsky, W. T. Vetterling, B. P. Flannery, \emph{Numerical
recipes in C}, 2nd ed., Cambridge University Press, 1992.
\bibitem{JT74}W. B. Jones and W. J. Thron, Math. Comp. \textbf{28}, p. 795 (1974).
\bibitem{KM97}A. H. Karp and P. Markstein, ACM TOMS \textbf{23}, p. 561 (1997). 
\bibitem{Ts88}Sh.~E. Tsimring, \emph{Handbook of special functions and definite
integrals: algorithms and programs for calculators}, Radio and communications
(publisher), Moscow, 1988 (in Russian).
\bibitem{McCabe74}J. H. McCabe, Math. Comp. \textbf{28}, p. 811 (1974).
\bibitem{Field77}D. A. Field, Math. Comp. \textbf{31}, p. 495 (1977).
\bibitem{Olver74}F. W. J. Olver, \emph{Asymptotics and special functions}, Academic
Press, 1974.\end{thebibliography}

\end{document}
