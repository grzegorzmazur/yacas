%% Macros needed for EYacas.
%% Jay Belanger (belanger@truman.edu)

%% Some of this was taken from Dan Dill's TeX/MMA (in modified
%% form, of course), and some was taken from other sources.  Some
%% I may even have written myself.

\def\Yacas{{\sffamily yacas\/}}
\def\yacasheadersize{\tiny}

%% An if to allow the lines to be turned on and off
\newif\ifyacaslines
\yacaslinestrue

%% Let's use some local macros

\catcode`\@=11\relax
\newdimen\yacasinputbaselineskip
\newdimen\yacasoutputbaselineskip
\newdimen\preyacasspace

%% Here are the values that the user may wish to change.
%% They can be changed here or in the main TeX file.

\font\yacasinputfont = cmtt10
\yacasinputbaselineskip = 10pt

\font\yacasoutputfont = cmtt10
\yacasoutputbaselineskip = 14pt

\preyacasspace = 0pt

%% First of all, set up some delimiters
%% The units are in terms of em and ex, so if the font size is
%% changed, the units will be changed accordingly

%% To make things line up, we'll need the size of "Initial"

\font\yac@sheaderfont=cmtt8  scaled 800
\setbox0=\hbox{\yac@sheaderfont\ Initial}
\newdimen\in@t
\in@t = \wd0
\advance\in@t by 1em

\def\sm@llsk{\vskip 1ex}

\def\yac@stop{
\vskip -1ex
\hbox to \hsize{\vrule depth 1ex  height .3pt width .4pt
                \vrule height .4pt depth 0pt width 1.1em 
                \lower .4ex \hbox{\yacasheadersize\ \Yacas} 
                \hrulefill 
                \vrule depth 1ex height .3pt width .4pt}
\vskip 1ex}


\def\yac@sinittop{
\vskip -1ex
\hbox to \hsize{\vrule depth 1ex  height .3pt width .4pt
                \vrule height .4pt depth 0pt width 1.1em 
                \lower .4ex \hbox{\yacasheadersize\ \Yacas} 
                \hrulefill
                \lower .4ex \hbox{\yac@sheaderfont\ Initial}
                \vrule height .4pt depth 0pt width 1em 
                \vrule depth 1ex height .3pt width .4pt}
\vskip 1ex}

\def\yac@sparttop#1{
\vskip -1ex
\hbox to \hsize{\vrule depth 1ex  height .3pt width .4pt
                \vrule height .4pt depth 0pt width 1.1em 
                \lower .4ex \hbox{\yacasheadersize\ \Yacas} 
                \hrulefill
                \lower .4ex \hbox{\yac@sheaderfont\ #1}
                \hrulefill
                \vrule height .4pt depth 0pt width \in@t
                \vrule depth 1ex height .3pt width .4pt}
\vskip 1ex}

\def\yac@sinitparttop#1{
\vskip -1ex
\hbox to \hsize{\vrule depth 1ex  height .3pt width .4pt
                \vrule height .4pt depth 0pt width 1.1em 
                \lower .4ex \hbox{\yacasheadersize\ \Yacas} 
                \hrulefill
                \lower .4ex \hbox{\yac@sheaderfont\ #1}
                \hrulefill
                \lower .4ex \hbox{\yac@sheaderfont\ Initial}
                \vrule height .4pt depth 0pt width 1em
                \vrule depth 1ex height .3pt width .4pt}
\vskip 1ex}


\def\yac@soutput{
        \vskip -1ex
        \hbox to \hsize{\hskip 1.5em
                        \vrule height .4pt depth 0pt width 3em
                        \lower .4ex \hbox{\yac@sheaderfont\ Output}
                        \hrulefill
                        \hskip 1.5em
                        }
        \vskip 1ex
}

\def\yac@stexoutput{
        \vskip -1ex
        \hbox to \hsize{\hskip 1.5em
                        \vrule height .4pt depth 0pt width 3em
                        \lower .4ex \hbox{\yac@sheaderfont\ \TeX\ Output}
                        \hrulefill
                        \hskip 1.5em
                        }
        \vskip 1ex
}

\def\yac@sgraphicsoutput{
        \vskip -1ex
        \hbox to \hsize{\hskip 1.5em
                        \vrule height .4pt depth 0pt width 3em
                        \lower .4ex \hbox{\yac@sheaderfont\ Graphics}
                        \hrulefill
                        \hskip 1.5em
                        }
        \vskip 1ex
}

\def\yac@sbottom{
\hbox to \hsize{\vrule depth 0ex height 1ex width .4pt
                \hrulefill 
                \vrule depth 0ex height 1ex width .4pt}
}


%% Next, it is necessary to have a verbatim environment for the Yacas
%% input and output

%% The verbatim environment for the input  is based mostly on the 
%% environment from "TeX for the Impatient" and Tim Morgan's verbtim2.tex

\def\deactivate{%
        \catcode`\{ = 12
        \catcode`\} = 12
        \catcode`\$ = 12
        \catcode`\& = 12
        \catcode`\# = 12
        \catcode`\% = 12
        \catcode`\~ = 12
        \catcode`\^ = 12
        \catcode`\_ = 12
        \catcode`\  = 12
}

{\catcode`\^^M=13
 {\catcode`\ =13\gdef\verbatimdefs{\def^^M{\ \par}\let =\ }}
 \gdef\verbatimgobble#1^^M{}}

\def\yac@sinputverbatim{
        \begingroup 
        \setlength{\parskip}{0pt}                     %% added by WMS
        \everypar{In> }
        \par
        \yacasinputfont
        \deactivate
        \verbatimdefs
        \catcode`\ =13 
        \catcode`\^^M=13
}

\let\endyac@sinputverbatim=\endgroup

%% The verbatim for Yacas output is taken from the TeXbook.  
%% It allows more symbols, but can't be used for the input
%% since the input can be ended with more than one delimiter.

\def\uncatcodespecials{\def\do##1{\catcode`##1=12 }\dospecials}

\chardef\active=13
\def\obeyspaces{\catcode`\ =\active}
{\catcode`\^^M=\active %
 \gdef\obeylines{\catcode`\^^M=\active \let^^M=\par}%
 \global\let^^M=\par} %


\def\setupyac@sverbatim{\yacasoutputfont
\obeylines \uncatcodespecials \obeyspaces}
{\obeyspaces\global\let =\ }


\def\yac@soutputverbatim{\par\begingroup
        \baselineskip = \yacasoutputbaselineskip
        \setupyac@sverbatim\doyac@sverbatim}      
{\catcode`\|=0 \catcode`\\=12 %
 |obeylines|gdef|doyac@sverbatim^^M#1\endyacas{#1|endgroup|ifyacaslines|yac@sbottom|fi|par|vskip 1.5ex|par|endgroup}}

%% Now we need to set up the actual environments.
%% We will need some "if"s to tell us where we are

\newif\ifyac@sinput
\newif\ifyac@soutput
\newif\ifyac@stexoutput
\newif\ifyac@sgraphics

\def\not@k{}

%% The top delimiter depends on whether or not it is an initial cell,
%% etc.

\def\yacas{
     \begingroup
     \par  \vskip \preyacasspace \par
     \baselineskip = \yacasinputbaselineskip
     \yac@sinputtrue
     \futurelet\nextchar\yac@s
}

\def\yac@s{
     \ifx\nextchar [%
         \let\next = \yac@sinit
     \else
         \ifx\nextchar <%
              \let\next = \yac@spart
         \else
               \let\next = \yac@sstart
         \fi
     \fi
     \next
}

\def\yac@sinit[#1]{
     \futurelet\nextchar%
     \yac@s@nit}

\def\yac@s@nit{
     \ifx\nextchar <%
          \let\next = \yac@sinitpart
     \else
          \let\next = \yac@sinitonly
     \fi
     \next
}


\def\yac@sinitonly{
     \par
     \vskip 1.5ex
     \par
     \ifyacaslines\yac@sinittop\fi
     \sm@llsk
     \yac@sinputverbatim
}


\def\yac@sstart{
     \par
     \vskip 1.5ex
     \par
     \ifyacaslines\yac@stop\fi
     \sm@llsk
     \yac@sinputverbatim
}


\def\yac@spart<#1:#2>{
     \par
     \vskip 1.5ex
     \par
     \def\testt@k{#2}
     \ifx\testt@k\not@k
         \def\yac@sparttext{Definition of package #1}
     \else
         \def\yac@sparttext{<#1:#2>}
     \fi
     \ifyacaslines\yac@sparttop\yac@sparttext\fi
     \futurelet\nextchar\yac@sp@rt
}

\def\yac@sp@rt{
     \sm@llsk
     \yac@sinputverbatim
}


\def\yac@sinitpart<#1:#2>{
     \par
     \vskip 1.5ex
     \par
     \def\testt@k{#2}
     \ifx\testt@k\not@k
         \def\yac@sparttext{Definition of package #1}
     \else
         \def\yac@sparttext{<#1:#2>}
     \fi
     \ifyacaslines\yac@sinitparttop\yac@sparttext\fi
     \futurelet\nextchar\yac@sp@rt
}


%% Now to take care of the various types of output

\def\output{
     \endyac@sinputverbatim
     \sm@llsk
     \yac@sinputfalse
     \yac@soutputtrue
     \ifyacaslines\yac@soutput\fi
%     \sm@llsk
     \yac@soutputverbatim
}      

\def\outputtex{
     \endyac@sinputverbatim
     \sm@llsk
     \yac@sinputfalse
     \yac@stexoutputtrue
     \ifyacaslines\yac@stexoutput\fi
     \sm@llsk
     \begingroup
     \def\par{}
     $$
}      

\def\@ndtexoutput{
    $$
    \endgroup
    \yac@stexoutputfalse
    \par
}

%% If the output is graphics, we'll need to load epsf.tex, but there's
%% no need to load it if there are no graphics, and there is no need
%% to load it more than once.

\newif\ifnogr@phics
\nogr@phicstrue
\def\outputgraphics{
    \endyac@sinputverbatim
    \sm@llsk
    \yac@sinputfalse
    \yac@sgraphicstrue
    \ifnogr@phics
        \input epsf.tex
        \nogr@phicsfalse
    \fi
    \ifyacaslines\yac@sgraphicsoutput\fi    
    \sm@llsk
}

\def\@ndoutputgraphics{}

%% We'll need to have a macro for the graphics

\def\mgraphics#1#2{
    \centerline{ \epsfxsize = #1 \epsfbox{#2}}
}


\def\endyacas{
     \ifyac@sinput 
         \endyac@sinputverbatim
         \yac@sinputfalse
      \else
         \ifyac@stexoutput
             \@ndtexoutput
         \else
            \ifyac@sgraphics
                \@ndoutputgraphics
            \fi
         \fi
     \fi
     \sm@llsk
     \ifyacaslines\yac@sbottom\fi
     \par
     \vskip 1.5 ex
     \par
     \endgroup
}

% Finally, to help with backslashes,
\def\bs{$\backslash$}

\catcode`\@=12\relax






